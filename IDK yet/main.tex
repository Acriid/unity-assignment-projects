\documentclass{article}
\usepackage{glossaries}

\makeglossaries

\newglossaryentry{reckless}
{
    name={reckless play},
    description={A way of playing in which the player does not think of the consequences
    of their actions.}
}

\newglossaryentry{cautious}
{
    name={cautious play},
    description ={A way of playing in which the player plays in order to avoid problems
    or consequences.}
}

\newglossaryentry{map}
{
    name = map,
    description = {The playable area in which the player can move around and interact
    with}
}



\begin{document}

\section{General description}
The prototype will be a 2D top down game in which the player will be given the task
of escaping the room that they were put in.

\section{Mechanics}


\subsection{Actions and Challenges}
The player will be able to move around in a 2D environment and interact with objects
within that environment. 
Some challenges the player may face include: Looking for objects to solve puzzles, 
hiding or running away from enemies.
The player will only be able to see parts of the \gls{map} which they explored



\subsection{Genre}
The genre of the game would be a Horror game.
Some sub-genres include: Puzzle, Narrative.

\subsection{Game examples}
The Stanley Parable: Different endings 
DarkWood: Fog of war



\section{Interrogation}

\subsection{Hypothesis}
When faced with limited information the player will engage in \gls{cautious} until they
get more information in which the player will engage in \gls{reckless}.

\subsection{Goals}
To make a field of view system in which the player can only see whats in their
immediate vicinity.
To make a system in which something in the game changes when the player completes the
game.

\section{Process}
User input code has been used from a previous project I did.
A generic tilemap was made in asseprite to use for a \gls{map}

\section{Player motivation design}
Make the player curious to keep them motivated. Maybe a door in which the player
can leave at any time, but their curiosity stops them.
\clearpage

\glsaddall
\printglossary

\end{document}